\documentclass[parskip=full]{scrartcl}

\usepackage[utf8]{inputenc}
\usepackage[T1]{fontenc}
\usepackage[ngerman]{babel} %ngermen = new german
\usepackage{hyperref}
\hypersetup{pdftitle= {PSE: Pflichtenheft,
		bookmarks=true,
}}
\usepackage{csquotes}
\usepackage{graphicx}
\usepackage{enumitem}
\usepackage[margin=2.5cm]{geometry}
	\title{KAnnect}
	\subtitle{Pflichtenheft}
	\subject{Praxis der Softwareentwicklung}
	\author{
		Amani
		\and Zeineb
		\and Mohammed
		\and Ibrahim
		\and Utku Erol
		\and Florian Giner
	}
	\date{Sommersemester 2018}
	
\begin{document}
	%\input{Deckblatt.tex}
	\maketitle
	\newpage
	\tableofcontents
	\newpage
	\setlength{\parindent}{0em}
	\setlength{\parskip}{0.5em}
	
	\section{Zielbestimmung}
	Die Nutzer sollen durch das Produkt in die Lage versetzt werden, leicht Informationen über Veranstaltungen rund um das Studium am
	KIT und um das Leben in Karlsruhe zu erlangen, sowie sich darüber zu unterhalten.
	
	\subsection{Musskriterien}
	
	\begin{itemize}
		\item Verwalten von Veranstaltungen
			\begin{itemize} 
				
				\item Erstellen, Löschen, Ändern, Suchen, Veranstaltungen abonnieren
				\item Hochladen von einem Bild als Hintergrund
				\item Ersteller hat Admin Rechte
			\end{itemize}
			
			\item Verwalten von Gruppen
				\begin{itemize}
					\item Erstellen, löschen, Ändern, Suchen, Speichern, Gruppen abonnieren
					\item Ersteller hat Admin Rechte
				\end{itemize}
			\item Verwalten von Kontakten
				\begin{itemize}
					\item Erstellen, löschen, Ändern, Suchen, Speichern, Kontakte abonnieren
					\item Erstellen des Benutzerkontos über Google Play
				\end{itemize}
			\item In-App Kommunikation
			\begin{itemize}
				\item versenden, erhalten von Nachrichten
				\item Veröffentlichung von Beiträgen von Nutzern
			\end{itemize}
			\item Beiträge im Feed darstellen
				\begin{itemize}
					\item von gefolgten Kontakten/Gruppen/Veranstaltungen
					\item anwenden von Filterkriterien
					\item chronologische Reihenfolge nach Veröffentlichungsdatum
				\end{itemize}
			\item Kommentieren von Beiträgen
				\begin{itemize}
					\item Speichern chronologisch , löschen, Ändern
				\end{itemize}	
	\end{itemize}
	
	\subsection{Wunschkriterien}
		\begin{itemize}[nosep]
			\item Sofortige Nachrichtenübermittlung
			\begin{itemize}
				\item erstellen und löschen von Konversationen und Gruppenkonversationen
				\item speichern der Nachrichten
				\item synchronisation aller Geräte
			\end{itemize}
			
			\item Karte von Karlsruhe
				\begin{itemize}
					\item zeigt alle Veranstaltungen des heutigen Tages und deren Ort
				\end{itemize}
			\item Verwalten von Geschäftskonten
				\begin{itemize}
					\item erstellen, löschen, Ändern, Suchen, Speichern
				\end{itemize}
			\item anbieten von Werbeflächen
				\begin{itemize}
					\item für Geschäftskonten
				\end{itemize}
			\item Sicherheit
				\begin{itemize}
					\item Kundendaten die über das Internet vermittelt \item werden sollen verschlüsselt werden
				\end{itemize}
			\item hochladen von Fotos und Videos
				\begin{itemize}
					\item für Nutzer und Geschäftskonten
				\end{itemize}
			\item Admin Rechte sollen weitergegeben werden können an andere Nutzer
			\item Chat mit zufälligen Nutzern
				\begin{itemize}
					\item Instant Messaging, Videochat und Sprach Chat
					\item mit anderen Nutzern die auch online sind, die diese Dienst auch nutzen wollen
				\end{itemize}
			\item Markieren  von Benutzern auf Beiträgen, in Gruppen und Veranstaltungen
			\item Chat mit zufälligen Nutzern
				\begin{itemize}
					\item Instant Messaging, Videochat und Sprach Chat
					\item mit anderen Nutzern die auch online sind, die diese Dienst auch nutzen wollen
				\end{itemize}
			\item Plattformunabhängigkeit
				\begin{itemize}
					\item das Produkt soll auf allen Plattformen erhältlich sein
					\item zB für iOS und Windows
				\end{itemize}
			\item Anmeldung mit anderen Services möglich
				\begin{itemize}
					\item zB Apple Account, Facebook Account, Twitter Account
				\end{itemize}
			\item Ändern der Sprache der App
				\begin{itemize}
					\item nicht nur Deutsch, sondern auch auf Englisch
				\end{itemize}
			\item Jobbörse für Unternehmen die Werkstudenten suchen
				\begin{itemize}
					\item erstellen, löschen, ändern, Suchen, Speichern von Stellenangeboten
				\end{itemize}
			\item private Gruppen
				\begin{itemize}
					\item erstellen, löschen, ändern und Speichern von privaten Gruppen
					\item nur eingeladene Nutzern können Gruppen beitreten	
				\end{itemize}
		\end{itemize}
		
	\subsection{Abgrenzungskriterien}
	\begin{itemize}
		\item Das Produkt soll nur auf Mobilfunk Telefone erhältlich sein.
		\item Kein Marketplace
	\end{itemize}
	
	\section{Produkteinsatz}
	Das Produkt soll eine Plattform bereitstellen, die das alltägliche Leben und Studieren in Karlsruhe nicht nur erleichtert, sondern auch verbindet. So sollen Nutzer leicht an Informationen über Veranstaltungen der jeweiligen Hochschule gelangen, wie auch an Informationen über Freizeitaktivitäten. Desweiteren bietet das Produkt noch die Möglichkeit sich leicht darüber zu unterhalten und Meinungen mit anderen Nutzern zu teilen.
	
	\subsection{Anwendungsbereiche}
		\begin{itemize}
			\item Akademischer Bereich
			\item Sozialer Bereich
		\end{itemize}
		
	\subsection{Zielgruppen}
		\begin{itemize}
			\item Studenten
			\item Mitarbeiter der Hochschulen in Karlsruhe
			\item Veranstalter
			% noch für private Personen?
		\end{itemize}
		
	\subsection{Betriebsbedingungen}
	
	
	\section{Produktumgebung}
	
	\subsection{Software}
	
	\subsection{Hardware}
	
	
	\section{Funktionale Anforderungen}
		\subsection{Freunde}
		\begin{itemize}[nosep]
			% Freund suchen
			\item[\textbf{FA10}]\textbf{Freund suchen}
			\newline\newline \textbf{Ziel:} Suche nach einem Freund mithilfe seines Namens
			\newline \textbf{Kategorie:} primär
			\newline \textbf{Vorbedingungen:} Nutzer muss ein Konto besitzen
			\newline \textbf{Nachbedingungen (Erfolg):} 
			\begin{enumerate}[nosep]
				\item Liste von Personen die den eingetippten Namen haben.
			\end{enumerate}
			\textbf{Nachbedingungen (Fehlschlag):} 
			\begin{enumerate}[nosep]
				\item Wenn keine Person diesen Namen besitzt, wird eine leere Seite mit einer Fehlernachricht angezeigt.
			\end{enumerate}  
			\textbf{Akteure:} Nutzer
			\newline \textbf{Auslösendes Ereignis:} Anfrage des Nutzers
			\newline \textbf{Beschreibung:}
				\begin{enumerate}[nosep]
					\item Im Suchfeld den Name und Vorname eintippen
					\item “Suche” klicken
					\item[3.a.] Eine Liste von Personen mit dem eingetippten Namen wird angezeigt (Erfolg)
					\item[3.b.] Sonst wird eine leere Seite mit einer Fehlernachricht angezeigt (Fehlschlag) \\
				\end{enumerate}
			
			% Freund abonnieren
			\item[\textbf{\large FA20}]\textbf{\large Freund abonnieren}
				\newline\newline \textbf{Ziel:} Beiträge des Freundes angezeigt zu bekommen
				\newline \textbf{Kategorie:} primär
				\newline \textbf{Vorbedingungen:} Nutzer und Freund müssen ein Konto besitzen
				\newline \textbf{Nachbedingungen (Erfolg):} 
				\begin{enumerate}[nosep]
					\item Freund wird zu der Abonnentenliste von dem Nutzer hinzugefügt
					\item “Abonnieren” Button wird zu “Nicht mehr abonnieren” Button
				\end{enumerate}
				\textbf{Nachbedingungen (Fehlschlag):}
				\newline \textbf{Akteure:} Nutzer
				\newline \textbf{Auslösendes Ereignis:} Anfrage des Nutzers
				\newline \textbf{Beschreibung:}
				\begin{enumerate}[nosep]
					\item Zu dem gewünschten Konto Profil navigieren
					\item Button “abonnieren” klicken
				\end{enumerate}
			
			% Freund nicht mehr abonnieren
			\item[\textbf{\large FA30}]\textbf{\large Freund nicht mehr abonnieren}
				\newline \textbf{Ziel:} Freund aus der Liste der Abonnenten löschen, damit seine Beiträge nicht mehr im Feed erscheinen
				\newline \textbf{Kategorie:} primär
				\newline \textbf{Vorbedingungen:} Freund muss schon abonniert sein
				\newline \textbf{Nachbedingungen (Erfolg):} 
				\begin{enumerate}[nosep]
					\item Freund wird aus der Liste der Abonnenten des Nutzers gelöscht
					\item “nicht mehr abonnieren” Button wird zu “Abonnieren” Button
				\end{enumerate}
				\textbf{Nachbedingungen (Fehlschlag):}
				\newline \textbf{Akteure:} Nutzer
				\newline \textbf{Auslösendes Ereignis:} Anfrage des Nutzers
				\newline \textbf{Beschreibung:}
				\begin{enumerate}[nosep]
					\item Zu der Profilseite des Freundes navigieren
					\item Auf Button “nicht mehr abonnieren” klicken
				\end{enumerate}
		\end{itemize}
		
		\subsection{Gruppe}
		\begin{itemize}[nosep]
			%Gruppe erstellen
			\item[\textbf{FA40}]\textbf{Gruppe erstellen}
			\newline \textbf{Ziel:} eine Gruppe erstellen
			\newline \textbf{Kategorie:} primär
			\newline \textbf{Vorbedingungen:} Nutzer muss angemeldet sein
			\newline \textbf{Nachbedingungen (Erfolg):} 
			\begin{enumerate}[nosep]
				\item Nutzer wird zur Seite der Gruppe weitergeleitet
				\item Gruppe wird zur Liste der gefolgten Gruppen des Erstellers hinzugefügt
				\item Gruppe wird für andere Nutzer sichtbar
			\end{enumerate}
			\textbf{Nachbedingungen (Fehlschlag):}
			\newline \textbf{Akteure:} Nutzer
			\newline \textbf{Auslösendes Ereignis:} Anfrage des Nutzers
			\newline \textbf{Beschreibung:}
			\begin{enumerate}[nosep]
				\item Zur Seite der Gruppen navigieren
				\item Auf Button “Gruppe Erstellen” klicken
				\item Ein Formular wird angezeigt , welches ausgefüllt werden muss
				\item Dann auf “Bestätigen” klicken
				\item Nutzer wird nun zur Seite der erstellten Gruppe weitergeleitet\\
			\end{enumerate}
			
			% Gruppe löschen
			\item[\textbf{FA50}]\textbf{Gruppe löschen}
						\newline \textbf{Ziel:} nicht mehr benötigte Gruppen löschen
						\newline \textbf{Kategorie:} primär
						\newline \textbf{Vorbedingungen:} Gruppe ist schon vorhanden
						\newline \textbf{Nachbedingungen (Erfolg):} 
						\begin{enumerate}[nosep]
							\item Gruppe wird aus System gelöscht
							\item Bestätigungsnachricht wird angezeigt
						\end{enumerate}
						\textbf{Nachbedingungen (Fehlschlag):}
						\newline \textbf{Akteure:} Ersteller der Gruppe
						\newline \textbf{Auslösendes Ereignis:} Anfrage des Erstellers der Gruppe
						\newline \textbf{Beschreibung:}
						\begin{enumerate}[nosep]
							\item Auf die Seite der zu löschende Gruppe navigieren
							\item Auf den Button “Löschen” klicken
							\item Aktion bestätigen\\
						\end{enumerate}
						
			% Gruppe beitreten
			\item[\textbf{FA60}]\textbf{Gruppe beitreten}
				\newline \textbf{Ziel:} Gruppenbeiträge angezeigt bekommen und neue Gruppenbeiträge veröffentlichen
				\newline \textbf{Kategorie:} primär
				\newline \textbf{Vorbedingungen:} Gruppe ist schon vorhanden
				\newline \textbf{Nachbedingungen (Erfolg):} 
					\begin{enumerate}[nosep]
						\item Nutzer wird zu der Abonnentenliste der Gruppe hinzugefügt und es wird eine Bestätigungsnachricht angezeigt
						\item ”Gruppe beitreten” Button wird zu “aus Gruppe austreten” 
					\end{enumerate}
				\textbf{Nachbedingungen (Fehlschlag):}
				\newline \textbf{Akteure:} Nutzer
				\newline \textbf{Auslösendes Ereignis:} Anfrage des Nutzers
				\newline \textbf{Beschreibung:}
					\begin{enumerate}[nosep]
						\item Auf die Seite der Gruppe navigieren
						\item Auf den Button “Beitreten” klicken
						\item Anzeigen einer Bestätigungsnachricht\\
					\end{enumerate}
									
			% Gruppe verlassen
			\item[\textbf{FA70}]\textbf{Gruppe verlassen}
				\newline \textbf{Ziel:} Gruppenbeiträge nicht mehr angezeigt zu bekommen
				\newline \textbf{Kategorie:} primär
				\newline \textbf{Vorbedingungen:} Benutzer ist Mitglied der vorhandenen Gruppe
				\newline \textbf{Nachbedingungen (Erfolg):} 
				\begin{enumerate}[nosep]
					\item Nutzer wird von der Abonnentenliste der Gruppe entfernt
					\item Eine Bestätigungsnachricht wird angezeigt
					\item ”Gruppe verlassen” Button wird zu “Gruppe beitreten” 
				\end{enumerate}
				\textbf{Nachbedingungen (Fehlschlag):}
				\newline \textbf{Akteure:} Mitglied der Gruppe
				\newline \textbf{Auslösendes Ereignis:} Anfrage eines Mitglieds der Gruppe
				\newline \textbf{Beschreibung:}
				\begin{enumerate}[nosep]
					\item Auf die Seite der Gruppe navigieren
					\item Auf den Button “Verlassen” klicken
					\item Anzeigen einer Bestätigungsnachricht\\
				\end{enumerate}
							
			% Beitrag in Gruppen
			\item[\textbf{FA80}]\textbf{Beitrag in Gruppe veröffentlichen}
				\newline \textbf{Ziel:} Kommunikation in der Gruppe
				\newline \textbf{Kategorie:} primär
				\newline \textbf{Vorbedingungen:} Nutzer ist Mitglied der Gruppe
				\newline \textbf{Nachbedingungen (Erfolg):} 
				\begin{enumerate}[nosep]
					\item Es wird ein neuer Beitrag in der Gruppe veröffentlicht 
				\end{enumerate}
				\textbf{Nachbedingungen (Fehlschlag):}
				\newline \textbf{Akteure:} Mitglied der Gruppe
				\newline \textbf{Auslösendes Ereignis:} Anfrage eines Mitglieds der Gruppe
				\newline \textbf{Beschreibung:}
				\begin{enumerate}[nosep]
					\item Auf die Seite der Gruppe navigieren
					\item Beitrag schreiben und auf den Button “Veröffentlichen” klicken
					\item  Beitrag erscheint im Feed der Gruppe\\
				\end{enumerate}
		
		% Beitrag in Gruppe löschen
		\item[\textbf{FA80}]\textbf{Beitrag in Gruppe löschen}
			\newline \textbf{Ziel:} Beiträge aus dem System löschen
			\newline \textbf{Kategorie:} primär
			\newline \textbf{Vorbedingungen:} Nutzer hat Beitrag verfasst oder ist Ersteller der Gruppe und Beitrag muss vorhanden sein
			\newline \textbf{Nachbedingungen (Erfolg):} 
			\begin{enumerate}[nosep]
				\item Beitrag wird aus dem System gelöscht
				\item Beitrag wird nicht mehr angezeigt 
			\end{enumerate}
			\textbf{Nachbedingungen (Fehlschlag):}
			\newline \textbf{Akteure:} Ersteller der Gruppe oder Verfasser des Beitrags
			\newline \textbf{Auslösendes Ereignis:} Anfrage des Erstellers der Gruppe oder des Verfassers des Beitrags
			\newline \textbf{Beschreibung:}
			\begin{enumerate}[nosep]
				\item Auf die Seite der Gruppe navigieren
				\item Zu dem zu löschenden Beitrag navigieren
				\item Auf den Button “Beitrag löschen” klicken
				\item Aktion bestätigen
				\item Anzeigen einer Bestätigungsnachricht
				\item Beitrag verschwindet aus dem Feed der Gruppe\\
			\end{enumerate}
						
		% Beitrag in Gruppe bearbeiten		
		\item[\textbf{FA80}]\textbf{Beitrag in Gruppe bearbeiten}
			\newline \textbf{Ziel:} ein veröffentlichter Beitrag in einer Gruppe ändern
			\newline \textbf{Kategorie:} primär
			\newline \textbf{Vorbedingungen:} Gruppe und Beitrag sind vorhanden, Nutzer muss Beitrag verfasst haben oder Ersteller der Gruppe sein
			\newline \textbf{Nachbedingungen (Erfolg):} 
			\begin{enumerate}[nosep]
				\item Beitrag wird geändert 
			\end{enumerate}
			\textbf{Nachbedingungen (Fehlschlag):}
			\newline \textbf{Akteure:} Ersteller des Beitrags
			\newline \textbf{Auslösendes Ereignis:} Anfrage des Erstellers des Beitrags
			\newline \textbf{Beschreibung:}
			\begin{enumerate}[nosep]
				\item Auf die Seite der Gruppe navigieren
				\item Zu dem Beitrag der geändert werden soll navigieren
				\item Auf “Beitrag bearbeiten” klicken
				\item Beitrag bearbeiten
				\item ”Änderungen speichern” klicken
				\item Geänderter Beitrag erscheint im Feed der Gruppe\\
			\end{enumerate}
						
		\end{itemize}
		
		\subsection{Veranstaltungen}
		\begin{itemize}[nosep]
			
			%Veranstaltung erstellen
			\item[\textbf{FA90}]\textbf{Veranstaltung erstellen}
				\newline \textbf{Ziel:} Bekanntmachung von Veranstaltungen, Werbemöglichkeit für eine Veranstaltung
				\newline \textbf{Kategorie:} primär
				\newline \textbf{Vorbedingungen:} Jeder Nutzer kann eine Veranstaltung erstellen
				\newline \textbf{Nachbedingungen (Erfolg):} 
				\begin{enumerate}[nosep]
					\item Veranstaltung wird in die Liste der Veranstaltungen hinzugefügt 
				\end{enumerate}
				\textbf{Nachbedingungen (Fehlschlag):}
				\newline \textbf{Akteure:} Nutzer
				\newline \textbf{Auslösendes Ereignis:} Anfrage des Nutzers
				\newline \textbf{Beschreibung:}
				\begin{enumerate}[nosep]
					\item zu der Seite der Veranstaltungen navigieren
					\item Auf den Button “Veranstaltung erstellen” klicken
					\item Formular wird angezeigt, das ausgefüllt werden muss
					\item Auf “Bestätigen” klicken
					\item Nutzer wird zur Seite der Veranstaltung weitergeleitet\\
				\end{enumerate}
				
			%Veranstaltung löschen
			\item[\textbf{FA100}]\textbf{Veranstaltung löschen}
				\newline \textbf{Ziel:} Veranstaltung aus dem System löschen
				\newline \textbf{Kategorie:} primär
				\newline \textbf{Vorbedingungen:} Veranstaltung muss vorhanden sein, Nutzer muss Ersteller sein
				\newline \textbf{Nachbedingungen (Erfolg):} 
				\begin{enumerate}[nosep]
					\item Veranstaltung wird aus Liste der Veranstaltungen gelöscht
					\item Veranstaltung wird aus dem System gelöscht
				\end{enumerate}
				\textbf{Nachbedingungen (Fehlschlag):}
				\newline \textbf{Akteure:}  Nutzer (Ersteller der Veranstaltung)
				\newline \textbf{Auslösendes Ereignis:} Anfrage des Nutzers
				\newline \textbf{Beschreibung:}
				\begin{enumerate}[nosep]
					\item Auf die Seite der Veranstaltung navigieren
					\item Auf den Button “Veranstaltung löschen” klicken
					\item Aktion bestätigen
					\item Anzeigen einer Bestätigungsnachricht\\
				\end{enumerate}
			
			%Veranstaltung abonnieren
			\item[\textbf{FA120}]\textbf{Veranstaltung abonnieren}
			\newline \textbf{Ziel:} Neuigkeiten und Informationen der Veranstaltung im Feed erhalten
			\newline \textbf{Kategorie:} primär
			\newline \textbf{Vorbedingungen:} Veranstaltung muss vorhanden sein, Nutzer muss ein Konto haben
			\newline \textbf{Nachbedingungen (Erfolg):} 
			\begin{enumerate}[nosep]
				\item Nutzer wird zu der Abonnentenliste der Veranstaltung hinzugefügt
				\item “Veranstaltung abonnieren” Button wird zu “ Veranstaltung nicht mehr abonnieren” 
			\end{enumerate}
			\textbf{Nachbedingungen (Fehlschlag):}
			\newline \textbf{Akteure:} Nutzer
			\newline \textbf{Auslösendes Ereignis:} Anfrage des Nutzers
			\newline \textbf{Beschreibung:}
			\begin{enumerate}[nosep]
				\item Auf die Seite der Veranstaltung navigieren
				\item Auf den Button “Veranstaltung abonnieren” klicken\\
			\end{enumerate}
						
						
			%Veranstaltung nicht mehr abonnieren
			\item[\textbf{FA130}]\textbf{Veranstaltung nicht mehr abonnieren}
			\newline \textbf{Ziel:} Veranstaltungs Neuigkeiten und Beiträge nicht mehr im Feed angezeigt zu bekommen
			\newline \textbf{Kategorie:} primär
			\newline \textbf{Vorbedingungen:} Nutzer muss die Veranstaltung abonniert haben
			\newline \textbf{Nachbedingungen (Erfolg):} 
			\begin{enumerate}[nosep]
				\item Nutzer wird aus der Abonnentenliste der Veranstaltung entfernt
				\item ”Veranstaltung nicht mehr abonnieren” Button wird zu “Veranstaltung abonnieren 
			\end{enumerate}
			\textbf{Nachbedingungen (Fehlschlag):}
			\newline \textbf{Akteure:} Nutzer
			\newline \textbf{Auslösendes Ereignis:} Anfrage des Nutzers
			\newline\newline \textbf{Beschreibung:}
			\begin{enumerate}[nosep]
				\item Zur Seite der Veranstaltung navigieren
				\item “Veranstaltung nicht mehr abonnieren” klicken
				\item  Aktion bestätigen\\
			\end{enumerate}
						
				
			%Veranstaltung bearbeiten
			\item[\textbf{FA140}]\textbf{Veranstaltung bearbeiten}
			\newline \textbf{Ziel:} Informationen über die Veranstaltung ändern
			\newline \textbf{Kategorie:} primär
			\newline \textbf{Vorbedingungen:} Nutzer muss Ersteller der Veranstaltung sein
			\newline \textbf{Nachbedingungen (Erfolg):} 
			\begin{enumerate}[nosep]
				\item Veranstaltung wird geändert 
			\end{enumerate}
			\textbf{Nachbedingungen (Fehlschlag):}
			\newline \textbf{Akteure:} Nutzer
			\newline \textbf{Auslösendes Ereignis:} Anfrage des Nutzers
			\newline\newline \textbf{Beschreibung:}
			\begin{enumerate}[nosep]
				\item Zur Seite der Veranstaltung navigieren
				\item “Veranstaltung bearbeiten” klicken
				\item  Formular wie beim Erstellen wird angezeigt und die Daten können geändert werden
				\item Bearbeitung bestätigen
				\item Nutzer wird zur Seite der Veranstaltung weitergeleitet\\
			\end{enumerate}
				
								
			%Beitrag in Veranstaltung veröffentlichen
			\item[\textbf{FA110}]\textbf{Beitrag in Veranstaltung veröffentlichen}
				\newline \textbf{Ziel:} Mitteilen von Informationen/Meinungen in einer Veranstaltung
				\newline \textbf{Kategorie:} primär
				\newline \textbf{Vorbedingungen:} Veranstaltung ist vorhanden, Veranstaltung ist abonniert vom Nutzer
				\newline \textbf{Nachbedingungen (Erfolg):} 
				\begin{enumerate}[nosep]
					\item Beitrag wird in der Veranstaltung veröffentlicht und auf dem Feed der Veranstaltung angezeigt
					\item Beitrag wird auf dem Feed der Nutzer angezeigt, die der Veranstaltung folgen
				\end{enumerate}
				\textbf{Nachbedingungen (Fehlschlag):}
				\newline \textbf{Akteure:} Nutzer
				\newline \textbf{Auslösendes Ereignis:} Anfrage des Nutzers
				\newline \textbf{Beschreibung:}
				\begin{enumerate}[nosep]
					\item Auf die Seite der Veranstaltung navigieren
					\item Auf den Button “Beitrag erstellen” klicken
					\item Beitrag erstellen
					\item ”Beitrag veröffentlichen” klicken\\
				\end{enumerate}
			
			%Beitrag in Veranstaltung löschen
			\item[\textbf{FA120}]\textbf{Beitrag in Veranstaltung löschen}	
				\newline \textbf{Ziel:} Beitrag aus dem System löschen
				\newline \textbf{Kategorie:} primär
				\newline \textbf{Vorbedingungen:} Nutzer hat den Beitrag verfasst oder Nutzer ist Ersteller der Veranstaltung
				\newline \textbf{Nachbedingungen (Erfolg):} 
				\begin{enumerate}[nosep]
					\item Beitrag wird aus dem System gelöscht
					\item Beitrag wird nicht mehr angezeigt 
				\end{enumerate}
				\textbf{Nachbedingungen (Fehlschlag):}
				\newline \textbf{Akteure:} Ersteller der Veranstaltung oder des Beitrags
				\newline \textbf{Auslösendes Ereignis:} Anfrage des Nutzers
				\newline \textbf{Beschreibung:}
				\begin{enumerate}[nosep]
					\item Auf die Seite der Veranstaltung navigieren
					\item Zu dem zu löschenden Beitrag navigieren
					\item Aktion bestätigen
					\item Anzeigen einer Bestätigungsnachricht
					\item Beitrag verschwindet aus dem Feed der Veranstaltung\\
				\end{enumerate}
			
			%Beitrag in Veranstaltung bearbeiten
			\item[\textbf{FA120}]\textbf{Beitrag in Veranstaltung bearbeiten}	
			\newline \textbf{Ziel:} Ein veröffentlichter Beitrag in einer Veranstaltung bearbeiten
			\newline \textbf{Kategorie:} primär
			\newline \textbf{Vorbedingungen:} Veranstaltung und Beitrag sind vorhanden, Nutzer muss den Beitrag verfasst haben
			\newline \textbf{Nachbedingungen (Erfolg):} 
			\begin{enumerate}[nosep]
				\item Beitrag wird geändert 
			\end{enumerate}
			\textbf{Nachbedingungen (Fehlschlag):}
			\newline \textbf{Akteure:} Ersteller des Beitrags
			\newline \textbf{Auslösendes Ereignis:} Anfrage des Nutzers
			\newline \textbf{Beschreibung:}
			\begin{enumerate}[nosep]
				\item Auf die Seite der Veranstaltung navigieren
				\item Zu dem Beitrag navigieren der geändert werden soll
				\item Auf "Beitrag bearbeiten" klicken
				\item Beitrag bearbeiten
				\item "Änderungen speichern" klicken
				\item Geänderter Beitrag erscheint im Feed der Veranstaltung\\
			\end{enumerate}
						


							
		\end{itemize}
		
		\subsection{Beiträge}
		\begin{itemize}[nosep]
			%Beitrag veröffentlichen
			\item[\textbf{FA150}]\textbf{Beitrag veröffentlichen}
			
			%Beitrag löschen
			\item[\textbf{FA160}]\textbf{Beitrag löschen}
			
			%Beitrag bearbeiten
			\item[\textbf{FA170}]\textbf{Beitrag bearbeiten}
			
			%Beitrag mit "Gefällt mir" markieren
			\item[\textbf{FA180}]\textbf{Beitrag mit "Gefällt mir" markieren}
						\newline \textbf{Ziel:} Zeigen das Beitrag dem Nutzer gefällt
						\newline \textbf{Kategorie:} Primär
						\newline \textbf{Vorbedingungen:} VNutzer muss angemeldet sein, Beitrag muss vorhanden sein
						\newline \textbf{Nachbedingungen (Erfolg):} 
						\begin{enumerate}[nosep]
							\item Anzahl der ‘Gefällt mir’ Angaben des Beitrags wird um eins erhöht
							\item “Gefällt mir” Button wird zu “Gefällt mir nicht mehr”
						\end{enumerate}
						\textbf{Nachbedingungen (Fehlschlag):}
						\newline \textbf{Akteure:} Nutzer
						\newline \textbf{Auslösendes Ereignis:} Anfrage des Nutzers
						\newline \textbf{Beschreibung:}
						\begin{enumerate}[nosep]
							\item Zu dem Beitrag navigieren
							\item Auf den Button “Gefällt Mir” klicken\\
						\end{enumerate}
					
					
						\item[\textbf{FA180}]\textbf{Beitrag "Gefällt mir nicht mehr"}
						\newline \textbf{Ziel:} Zeigen das Beitrag dem Nutzer nicht mehr gefällt
						\newline \textbf{Kategorie:} primär
						\newline \textbf{Vorbedingungen:} Beitrag muss vorhanden sein und von dem Nutzer mit "Gefällt mir" markiert sein
						\newline \textbf{Nachbedingungen (Erfolg):} 
						\begin{enumerate}[nosep]
							\item “Gefällt mir nicht mehr” Button wird zu “Gefällt mir”
							\item Anzahl der ‘Gefällt mir’ Angaben des Beitrags wird um eins  verringert 
						\end{enumerate}
						\textbf{Nachbedingungen (Fehlschlag):}
						\newline \textbf{Akteure:} Nutzer
						\newline \textbf{Auslösendes Ereignis:} Anfrage des Nutzers
						\newline \textbf{Beschreibung:}
						\begin{enumerate}[nosep]
							\item Zu dem Beitrag navigieren
							\item Auf den Button “Gefällt mir nicht mehr” klicken\\
						\end{enumerate}
						
				
			%Beiträge kommentieren
			\item[\textbf{FA190}]\textbf{Beiträge kommentieren}
				\newline \textbf{Ziel:} Meinungen/Informationen/Reaktion zu einem Beitrag äußern
				\newline \textbf{Kategorie:} Primär
				\newline \textbf{Vorbedingungen:} Nutzer muss eingeloggt sein, Beitrag muss vorhanden sein
				\newline \textbf{Nachbedingungen (Erfolg):} 
				\begin{enumerate}[nosep]
					\item Kommentar wird gespeichert
					\item Kommentar wird unter dem Beitrag angezeigt 
				\end{enumerate}
				\textbf{Nachbedingungen (Fehlschlag):}
				\newline \textbf{Akteure:} Nutzer
				\newline \textbf{Auslösendes Ereignis:} Anfrage des Nutzers
				\newline \textbf{Beschreibung:}
				\begin{enumerate}[nosep]
					\item Zu dem Beitrag navigieren
					\item Auf den ‘Kommentieren’ Button klicken
					\item Kommentar schreiben
					\item “Kommentar veröffentlichen” klicken\\
				\end{enumerate}
									
		\end{itemize}
		
		\subsection{Konto}
		\begin{itemize}[nosep]
			
			%Konto erstellen
			\item[\textbf{FA200}]\textbf{Konto erstellen}
							\newline \textbf{Ziel:} KAnnect zu benutzen
							\newline \textbf{Kategorie:} Primär
							\newline \textbf{Vorbedingungen:} Nutzer muss ein Google Konto besitzen und es darf nicht schon ein Konto registriert sein über diesem Google Konto
							\newline \textbf{Nachbedingungen (Erfolg):} 
							\begin{enumerate}[nosep]
								\item Konto wird erstellt und gespeichert
								\item Nutzer wird zu seinem Feed weitergeleitet 
							\end{enumerate}
							\textbf{Nachbedingungen (Fehlschlag):} Eine Fehlernachricht wird angezeigt
							\newline \textbf{Akteure:} Nutzer
							\newline \textbf{Auslösendes Ereignis:} Anfrage des Nutzers
							\newline \textbf{Beschreibung:}
							\begin{enumerate}[nosep]
								\item App herunterladen
								\item App öffnen
								\item Auf den “Registrieren” Button klicken
								\item Mit Google-Konto anmelden und bestätigen\\
							\end{enumerate}
		
			% Konto löschen			
			\item[\textbf{FA210}]\textbf{Konto löschen}
							\newline \textbf{Ziel:} App nicht mehr benutzen, alle Daten über Nutzer zu löschen
							\newline \textbf{Kategorie:} Primär
							\newline \textbf{Vorbedingungen:} Nutzer muss ein Konto haben und angemeldet sein
							\newline \textbf{Nachbedingungen (Erfolg):} 
							\begin{enumerate}[nosep]
								\item Nutzerdaten werden von der App gelöscht
								\item Nutzer wird abgemeldet 
							\end{enumerate}
							\textbf{Nachbedingungen (Fehlschlag):}
							\newline \textbf{Akteure:} Nutzer
							\newline \textbf{Auslösendes Ereignis:} Anfrage des Nutzers
							\newline \textbf{Beschreibung:}
							\begin{enumerate}[nosep]
								\item Auf das eigene Profil navigieren
								\item Auf “Einstellungen” klicken
								\item Auf “Konto löschen” klicken
								\item Aktion bestätigen\\
							\end{enumerate}
			
			% anmelden				
			\item[\textbf{FA220}]\textbf{anmelden}
							\newline \textbf{Ziel:} App benutzen
							\newline \textbf{Kategorie:} Primär
							\newline \textbf{Vorbedingungen:} Nutzer muss registriert sein
							\newline \textbf{Nachbedingungen (Erfolg):} 
							\begin{enumerate}[nosep]
								\item Nutzer wird zu seinem Feed weitergeleitet 
							\end{enumerate}
							\textbf{Nachbedingungen (Fehlschlag):}
							\newline \textbf{Akteure:} Nutzer
							\newline \textbf{Auslösendes Ereignis:} Anfrage des Nutzers
							\newline \textbf{Beschreibung:}
							\begin{enumerate}[nosep]
								\item App öffnen
								\item Auf den Button “Anmelden” klicken
								\item Mit Google-Konto anmelden\\
							\end{enumerate}
						
			% abmelden	
			\item[\textbf{FA230}]\textbf{abmelden}
							\newline \textbf{Ziel:} App nicht mehr benutzen oder sich mit einem anderen Kono anmelden
							\newline \textbf{Kategorie:} Primär
							\newline \textbf{Vorbedingungen:} Nutzer ist angemeldet
							\newline \textbf{Nachbedingungen (Erfolg):} 
							\begin{enumerate}[nosep]
								\item Nutzer wird abgemeldet
								\item Nutzer wird zur Anmeldeseite weitergeleitet 
							\end{enumerate}
							\textbf{Nachbedingungen (Fehlschlag):}
							\newline \textbf{Akteure:} Nutzer
							\newline \textbf{Auslösendes Ereignis:} Anfrage des Nutzers
							\newline \textbf{Beschreibung:}
							\begin{enumerate}[nosep]
								\item Zu dem “Abmelden” Button navigieren\\
							\end{enumerate}
							
		\end{itemize}
		
		\subsection{Nachrichten}
		\begin{itemize}[nosep]
			\item[\textbf{FA240}]\textbf{Nachrichten schicken}
				\newline \textbf{Ziel:} Mit anderen Nutzern zu kommunizieren
				\newline \textbf{Kategorie:} Primär
				\newline \textbf{Vorbedingungen:} Nutzer ist angemeldet und Empfänger muss ein Konto haben
				\newline \textbf{Nachbedingungen (Erfolg):} 
				\begin{enumerate}[nosep]
					\item Nachricht wird versendet
					\item Nachricht wird zu dem Ordner “Gesendete Nachrichten” hinzugefügt
				\end{enumerate}
				\textbf{Nachbedingungen (Fehlschlag):}
				\newline \textbf{Akteure:} Nutzer
				\newline \textbf{Auslösendes Ereignis:} Anfrage des Nutzers
				\newline \textbf{Beschreibung:}
				\begin{enumerate}[nosep]
					\item Möglichkeit
						\begin{enumerate}[nosep]
							\item Zum Profil des Empfängers navigieren
							\item “Nachricht schicken” klicken
							\item Nachricht verfassen
							\item Auf den Button “Schicken” klicken
						\end{enumerate}
					\item Möglichkeit
						\begin{enumerate}[nosep]
							\item Zum Briefkasten navigieren
							\item “Nachricht schicken” klicken
							\item Empfänger auswählen
							\item Nachricht schreiben
							\item Auf den Button “Schicken” klicken
						\end{enumerate}
				\end{enumerate}
										
			\item[\textbf{FA250}]\textbf{ Nachrichten empfangen}
				\newline \textbf{Ziel:} Überprüfen ob neue Nachrichten im Briefkasten sind
				\newline \textbf{Kategorie:} Primär
				\newline \textbf{Vorbedingungen:} Nutzer ist angemeldet
				\newline \textbf{Nachbedingungen (Erfolg):} 
				\begin{enumerate}[nosep]
					\item neue Nachrichten werden im Briefkasten angezeigt
					\item Neue Nachrichten können angeklickt und gelesen werden 
				\end{enumerate}
				\textbf{Nachbedingungen (Fehlschlag):}
				\newline \textbf{Akteure:} Nutzer
				\newline \textbf{Auslösendes Ereignis:} Anfrage des Nutzers
				\newline \textbf{Beschreibung:}
				\begin{enumerate}[nosep]
					\item Zum “Briefkasten” navigieren
					\item Auf die neue Nachricht klicken
					\item Nachricht wird geöffnet\\
				\end{enumerate}
										
		\end{itemize}
		
		\subsection{Feed}
		\begin{itemize}[nosep]
			
			% Persönlicher Feed
			\item[\textbf{FA260}]\textbf{persönlicher Feed}
							\newline \textbf{Ziel:} Beiträge von abonnierten Veranstaltungen/Freunde und beigetretenen Gruppen zu sehen
							\newline \textbf{Kategorie:} Primär
							\newline \textbf{Vorbedingungen:} Nutzer ist angemeldet
							\newline \textbf{Nachbedingungen (Erfolg):} 
							\begin{enumerate}[nosep]
								\item Beiträge werden im Feed angezeigt 
							\end{enumerate}
							\textbf{Nachbedingungen (Fehlschlag):}
							\newline \textbf{Akteure:} Nutzer
							\newline \textbf{Auslösendes Ereignis:} Anfrage des Nutzers
							\newline \textbf{Beschreibung:}
							\begin{enumerate}[nosep]
								\item Zum Feed navigieren\\
							\end{enumerate}
		\end{itemize}
	\section{Produktdaten}
	\begin{itemize}
		\item[\textbf{PD10}] \textbf{Nutzer}
			
		\item[\textbf{PD20}] \textbf{Kategorie}
		\item[\textbf{PD21}] \textbf{Unterkategorie}
		\item[\textbf{PD30}] \textbf{Gruppe}
		\item[\textbf{PD40}] \textbf{Veranstaltung}
		\item[\textbf{PD50}] \textbf{Beitrag}
		\item[\textbf{PD60}] \textbf{Nachricht}
	\end{itemize}
	
	\section{Nichfunktionale Anforderungen}
	
	\section{Globale Testfälle}
	
	\section{Systemmodelle}
	\subsection{Szenarien}
		\subsubsection{Registrierung}
		Bob möchte sich über die Veranstaltungen an den Hochschulen in Karlsruhe informieren, Beiträge von Gruppen sehen und kommentieren. Deswegen geht er auf Google-Play, um eine passende App zu finden. Er findet KAnnect und ist sofort überzeugt. Er klickt auf den "Herunterladen" Button. Nach der Installation öffnet Bob die App und klickt auf “Neues Konto erstellen”. Bob registriert sich erfolgreich mit seinem Google-Konto. Dann wird er zu seinem Feed weitergeleitet.
		
		\subsubsection{Freunde}
		Alex ist ein neuer Student in Karlsruhe.  Während der O-Phase hat er einen Kommilitonen kennengelernt, namens Bob. Um die Kommunikation zu erleichtern, hat Bob Alex empfohlen, die neue App  KAnnect zu installieren und ihn zu abonnieren. Zu Hause gibt Alex den Namen Bob im Suchfeld ein. Danach klickt er auf den Button ‘Abonnieren’. Dadurch bekommt er jetzt alle Beiträge von Bob auf seinem Feed angezeigt. Am Wochenende hat Alex Lust mit Bob feiern zu gehen, deswegen hat er ihm eine Nachricht über KAnnect geschrieben. Nach einigen Minuten öffnet er seine Mailbox und sieht, dass er eine Antwort von Bob erhalten hat. Der bestätigt das Treffen in einer Stunde am Schloss. Nach zwei Stunden warten am Schloss, beschließt Alex, sehr erzürnt, heimzugehen. Daheim angelangt geht Alex auf KAnnect und öffnet Bobs Profil und klickt “nicht mehr abonnieren”.
		
		\subsubsection{Gruppen}
		Die neue Hochschulgruppe „Technologien für Studenten“ möchte ihre Neuigkeiten den Studenten präsentieren. Alex, ein Verantwortlicher der Gruppe meldet sich mit seinem Konto an. Er drückt in der App auf “Gruppe erstellen” und muss nun ein Formular ausfüllen. Nachdem er das gemacht hat, drückt er “Bestätigen” und die Gruppe wird zu der Liste der gefolgten Gruppen hinzugefügt %Link zu GUI
		Bob ist ein Student und interessiert sich für diese Gruppe. Er meldet sich in der App an und gibt den Namen der Gruppe in das Suchfeld ein. Nun wird ihm die gesuchte Gruppe angezeigt. Er klickt auf „Gruppe beitreten“ und kann nun alle Beiträge in der Gruppe sehen. Diese werden ihm nun auch in seinem Feed angezeigt. Bob fragt sich, wie man Mitglied der Hochschulgruppe werden kann, weshalb er einen Beitrag in der Gruppe veröffentlicht. Nach rund fünf Minuten sieht Alex den Beitrag von Bob und erklärt den Bewerbungsprozess, indem er den Beitrag kommentiert. Da Alex davon ausgeht, dass noch mehr Leute dieselbe Frage haben, verfasst er einen Beitrag über den Bewerbungsprozess und veröffentlicht diesen in der Gruppe. Nach der Veröffentlichung bemerkt Alex, dass er einen veralteten Prozess erläutert hat. Deswegen klickt er auf „Beitrag löschen“.
		Da Bob nach einer Woche immer noch keine richtige Antwort erhalten hat, tritt er aus der Gruppe aus.
		Bob bemerkt, dass seine Gruppe “O-Phase 2014” nicht mehr aktiv ist, und will sie löschen. Er geht auf der Seite der Gruppe und klickt auf „Gruppe löschen“. Nachdem er “Bestätigen” klickt, wird die Gruppe gelöscht.
		
		\subsubsection{Veranstaltungen}
		Manfred Maier ist ein Student in Karlsruhe. Die Klausurenphase ist endlich vorbei. Er hat Lust heute Nachmittag etwas zu machen, aber weiß nicht genau was. Er fragt seinen besten Freund Siegfried, was heute in Karlsruhe abgeht. Leider kann er auch nichts Besonderes vorschlagen und sagt, dass sie KAnnect checken sollen. Manfred denkt, dass es eine sehr gute Idee ist, und öffnet die App.
		Auf der Hauptseite klickt er auf den Button "Veranstaltungen" und befindet sich nun auf einer neuen Seite, auf der alle Veranstaltungen nach Datum sortiert sind. Oben sind die Veranstaltungen, die heute stattfinden werden, weiter unten die der kommenden Tage und Wochen.
		Da ihm so viele Veranstaltungen angezeigt werden, will er die Veranstaltungen filtern. Er klickt auf "eine Veranstaltungskategorie wählen" und sieht alle Kategorien wie "Sport" "Studium" "Nightclubs" "Bars \& Pubs" "Essen" usw. 
		Dann wählt er "Bars \& Pubs" und die Seite aktualisiert sich. Nun werden nur die Veranstaltungen angezeigt, die etwas mit der Kategorie zu tun haben.
		Er sieht, dass eine Bar heute von 17 bis 19 Uhr Happy Hour mit Live Musik anbietet. Er klickt auf die Veranstaltung, um weitere Details zu sehen. Er befindet sich nun auf der Seite der Veranstaltung. Dort werden ihm veröffentlichte Beiträge der Veranstaltung angezeigt. Nach einer kurzen Diskussion, entscheiden sich Manfred und Siegfried in diese Bar zu gehen. Um neue Informationen über die Veranstaltung in seinem Feed angezeigt zu bekommen, klickt Manfred auf "Veranstaltung abonnieren". Außerdem wollen sie später noch zu irgendeiner Party gehen. Ganz oben auf der Veranstaltungsseite ist ein Suchfeld "Veranstaltungen suchen". Da gibt er "Party" ein und die Seite aktualisiert sich. Nun werden ihm alle Veranstaltungen die das Schlüsselwort "Party" haben angezeigt. Er mag diese Veranstaltungen nicht und entscheidet sich, eine eigene Party zu schmeißen. Er redet mit Siegfried und sie machen die Partyvorbereitungen zusammen. Sie wollen jeden einladen, der mitfeiern will. Er geht auf "Erstelle neue Veranstaltung" und bekommt ein Formular von der App: 
		
		%link zu gui
		
		Nachdem er es fertig ausgefüllt hat, bestätigt er das Erstellen der Veranstaltung und wird zu der Seite seiner Veranstaltung weitergeleitet. Er kontrolliert die Seite noch mal und sieht, dass er eine falsche Adresse eingegeben hat. Er klickt auf "Veranstaltung bearbeiten" und wird wieder zu dem Formular geleitet.Nun korrigiert er seinen Fehler und bestätigt seine Änderungen und kommt wieder zurück zu der Veranstaltungsseite.
		
		Eine Woche später möchte er wieder so eine Party schmeißen und erstellt wieder eine neue Veranstaltung in der App. Am nächsten Tag bekommt er seine Klausurergebnisse und ihm gefallen seine Noten überhaupt nicht. Deswegen möchte er sich mehr auf sein Studium konzentrieren und sagt die Party ab. Er öffnet KAnnect und navigiert auf die Seite seiner erstellten Veranstaltung. Dann klickt er auf “Veranstaltung löschen”.
		
		\subsubsection{Feed}
		Bob öffnet KAnnect und möchte sein Feed durchschauen. Er sieht die Beiträge seiner Freunde, seiner abonnierten Veranstaltungen und seiner gefolgten Gruppen in chronologische Reihenfolge.
		Bob will auf ein Beitrag reagieren , er klickt “Gefällt mir”, weil ein Freund von ihm was Lustiges gepostet hat. Er ist die 16.te Person, die den Beitrag mit “Gefällt mir” markiert hat. Außerdem kommentiert er den Beitrag, daraufhin merkt er, dass auch diese chronologisch sortiert sind.
		
	\subsection{Anwendungsfälle}
	
	\section{Grafische Benutzeroberfläche}
	
	\section{Durchführbarkeitsanalyse}
	
	\subsection{Technische Durchführbarkeit}
	\subsection{Personelle Durchführbarkeit}
	\subsection{Alternative Lösungsvorschläge}
	\subsection{Risiken}
	
	
		
	
	
\end{document}